\begin{resumo} 

O RSA é um algoritmo de criptografia assimétrica amplamente utilizado que permite garantir o estabelecimento de comunicações seguras em ambientes abertos como a internet. O objetivo desse artigo é descrever conceitualmente o funcionamento do algotirmo RSA e alguns pontos de seu embasamento matemático, a implementação realizada e sua complexidade, com o foco especial na análise de performance dos processos de geração e de quebra de chave. Dentro dos resultados atingidos é possível identificar que a segurança de sistemas de criptografia está baseada primordialmente na garantia de que a fatoração de grandes chaves demanda um tempo relativamente alto, mesmo para grande capacidade computacional. Nota-se também que a utilização de testes probabilísticos como os de Fermat e Miller-Rabin, assim como heurísticas como Pollard-Rho, permitem que se tenha ganho de performance, mas sem comprometer a segurança do sistema de criptografia.
\newline
\newline
Palavras-chave: Criptografia, RSA, Pollard Rho, Fermat, Miller-Rabin.

\end{resumo}