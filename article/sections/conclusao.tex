

O processo de criptografia tem um embasamento matemático muito forte, desenvolvido e aperfeiçoado durante décadas por diversos estudiosos e pesquisadores.

Nos resultados obtidos a partir da execução dos algoritmos desenvolvidos foi possível constatar que, como descrito na bibliografia pesquisada, o processo de geração de chaves, assim como as técnicas de criptografia e descriptografia executam em tempo polinomial, o que é considerado eficiente. A fatoraçao de números, porém, apresenta complexidade exponencial, tornando-se computacionalmente inviável para inteiros muito grandes. Mesmo com o aumento da capacidade computacional não existem técnicas atualmente que permitam a quebra da criptografia RSA para chaves com um grande número de bits (por exemplo 2048 bits).

Percebe-se ainda que chaves pequenas podem ser facilmente quebradas, mesmo quando testados em computadores com baixo poder de processamento, o que enfatiza a importância da escolha de chaves de tamanho adequado, sendo esse o fator mais importante para a segurança no processo de criptografia.

Como abordado por Burnett (2002), a criptografia não é a única ferramenta necessária para assegurar a segurança dos dados, nem consegue resolver todos os problemas de segurança pois ela é apenas um instrumento entre vários outros. Além disso, a criptografia não é à prova de falhas. Toda criptografia pode ser quebrada e, sobretudo, se for implementada incorretamente, ela não agrega nenhuma segurança real.
