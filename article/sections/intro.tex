A criptografia costuma ser definida como a arte de escrever em cifra ou em código, de modo a permitir que somente quem conheça o código possa ler a mensagem \cite{marcacini:10}. Com o crescente uso de dispositivos conectados a internet, como computadores, smartphones, equipamentos industriais (IIOT) e domésticos (IOT), que se tornam cada vez mais presentes na vida diária, a necessidade da segurança eletrônica torna-se ainda mais crítica. A utilização eficaz de técnicas criptográficas está no núcleo de várias dessas estratégias de gerenciamento de riscos de roubo de informações. \cite{burnett:02}


O sistema de criptografia de chave pública RSA se baseia na diferença drástica entre a facilidade de encontrar números primos grandes e a dificuldade de fatorar o produto de dois números primos grandes \cite{cormen:02}. A patente do algoritmo RSA expirou em setembro de 2000, o que permite que qualquer pessoa possa criar implementações desse algoritmo e, assim, aumentar ainda mais a utilização do RSA \cite{burnett:02}.


Neste artigo serão apresentados brevemente os conceitos da criptografia, com ênfase na criptografia RSA. Para melhor entendimento foram detalhados alguns procedimentos fundamentais para sua construção, como a geração de números primos grandes e o processo de quebra de chaves por força bruta.

O artigo está divido em seções, sendo que a primeira apresenta a criptografia e o sistema de geração de chaves, seguindo com a análise de sua implementação e da performance de algoritmos para geração e quebra de chaves e por fim a conclusão final.
