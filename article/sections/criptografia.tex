Quando há necessidade de se proteger informações sigilosas utiliza-se o processo de criptografia para codificá-las de forma que não possam ser facilmente interpretadas. A leitura das informações criptografadas precisa passar por um novo processo de conversão, que é descriptografia, para que se torne novamente legível.

Um tipo muito comum de criptografia é chamado de chave simétrica. Nessa abordagem, um algoritmo utiliza uma chave para codificar informações em algo que se parece com um conjunto aleatório. O algoritmo, então, se utiliza da mesma chave para recuperar os dados originais \cite{burnett:02}.

Um sistema de criptografia de chave pública pode ser usado para codificar mensagens enviadas entre dois participantes de uma comunicação, de forma que um intruso que escute as mensagens codificadas não possa decodificá-las. Um sistema de criptografia de chave pública também permite que um dos participantes acrescente uma "assinatura digital"\ impossível de forjar ao final de uma mensagem eletrônica. Tal assinatura é a versão eletrônica de uma assinatura manuscrita em um documento de papel. Ela pode ser conferida com facilidade por qualquer pessoa, não pode ser forjada por ninguém, e ainda perde sua validade se qualquer bit da mensagem for alterado. Portanto, ela fornece autenticação, tanto da identidade do signatário quanto do conteúdo da mensagem assinada \cite{cormen:02}.

No sistema de criptografia RSA um usuário cria e torna pública uma chave baseada em dois números primos grandes, os quais são mantidos em segredo, ao mesmo tempo que cria uma chave privada, também mantida em segredo. Qualquer pessoa pode usar a chave pública para codificar uma mensagem e enviá-la ao destinatário que, em posse da chave privada, consegue descriptografá-la facilmente.

Cormen (2002) descreve os procedimentos a seguir, exemplificados pelos autores:
\begin{enumerate}
    \item Selecionar dois números primos grandes $p$ e $q$, sendo $p \neq q$, tal como 19 e 31;
    \item Calcular $n$ pela equação $n=p*q$, por exemplo $n$ = 589;
    \item Calcular o totiente de Euler, dado por $\phi(n) = (p-1)*(q-1)$, nesse caso $\phi(n) = 540$;
    \item Selecionar um inteiro ímpar pequeno $e$, tal que seja primo relativo de $\phi(n)$, satisfazendo a condição $MDC(e,\phi(n)) = 1$, nesse exemplo $e = 59$;
    \item Calcular $d$ como o inverso multiplicativo de $e$, utilizando o Algoritmo de Euclides estendido, que nesse exemplo resultaria em $d = 119$;
    \item O par formado por $P = (e,n)$ compõe a chave pública do RSA  e o par formado por $S = (d,n)$ a chave privada.
\end{enumerate}

O tamanho da chave é o fator mais importante para garantia da segurança no processo de criptografia. A Infraestrutura de Chaves Públicas do Brasil – ICP-Brasil recomenda o uso do algoritmo RSA como padrão para geração de chaves criptográficas de, no mínimo, 2048 bits \cite{icp:09}. A segurança do criptossistema RSA é fortemente vinculada à fatoração de $n$, que pode revelar os valores de $p$ e $q$. É devido ao grande esforço que demanda realizar essa fatoração e quebra da chave que se entende que o sistema de criptografia RSA é seguro. \cite{goodrich:13}.