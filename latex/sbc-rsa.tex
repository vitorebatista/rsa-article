% Dúvidas do formato Latex?
% http://www.docs.is.ed.ac.uk/skills/documents/3722/3722-2014.pdf

\documentclass[12pt]{article}
\usepackage{sbc-rsa}
\usepackage{graphicx,url}
\usepackage[brazil]{babel}   
%\usepackage[latin1]{inputenc}  
\usepackage[utf8]{inputenc}  
% UTF-8 encoding is recommended by ShareLaTex

\sloppy

\title{Implementação e Análise de Complexidade do Sistema de Criptografia de Chave Pública RSA}

\author{Felipe Nathan Welter\inst{1}, Vitor Emanuel Batista\inst{1} }

\address{Centro de Ciências Tecnológicas -- Universidade do Estado de Santa Catarina
  (UDESC)\\
  89.219-710 -- Joinville -- SC -- Brazil
  \email{\{felipenwelter,vitorebatista\}@gmail.com}
}

\begin{document} 
\maketitle
\begin{abstract}
  bla bla bla in English
\end{abstract}
\begin{resumo} 

O RSA é um algoritmo de criptografia assimétrica amplamente utilizado que permite garantir o estabelecimento de comunicações seguras em ambientes abertos como a internet. O objetivo desse artigo é descrever conceitualmente o funcionamento do algotirmo RSA e alguns pontos de seu embasamento matemático, a implementação realizada e sua complexidade, com o foco especial na análise de performance dos processos de geração e de quebra de chave. Dentro dos resultados atingidos é possível identificar que a segurança de sistemas de criptografia está baseada primordialmente na garantia de que a fatoração de grandes chaves demanda um tempo relativamente alto, mesmo para grande capacidade computacional. Nota-se também que a utilização de testes probabilísticos como os de Fermat e Miller-Rabin, assim como heurísticas como Pollard-Rho, permitem que se tenha ganho de performance, mas sem comprometer a segurança do sistema de criptografia.
\newline
\newline
Palavras-chave: Criptografia, RSA, Pollard Rho, Fermat, Miller-Rabin.

\end{resumo}


\section{Introdução}

    \textbf{Sugestão de tópicos}
    \newline
    \textit{
    * [ok] O que é o básico de criptografia \newline
    * [ok] O por que da criptografia (aumento de computadores, celulares, etc)\newline
    * [pendente] Tipos de criptografia e descriptografia \newline
    * O mais utilizado é o RSA e o motivo \newline
    * [ok] O que será abordado no artigo \newline
    * [ok] Como está dividido o artigo (tópicos) \newline
    }

A criptografia costuma ser definida como a arte de escrever em cifra ou em código, de modo a permitir que somente quem conheça o código possa ler a mensagem \cite{marcacini:10}. Com o crescente uso de dispositivos conectados a internet, como computadores, smartphones, equipamentos industriais (IIOT) e domésticos (IOT), tornam-se cada vez mais presentes na vida diária. Desta forma, a necessidade da segurança eletrônica torna-se ainda mais crítica. A utilização eficaz de técnicas criptográficas está no núcleo de várias dessas estratégias de gerenciamento de riscos de roubo de informações. \cite{burnett:02}


O sistema de criptografia de chave pública RSA se baseia na diferença drástica entre a facilidade de encontrar números primos grandes e a dificuldade de fatorar o produto de dois números primos grandes \cite{cormen:02}. A patente do algoritmo RSA expirou em setembro de 2000, o que permite que qualquer pessoa possa criar implementações desse algoritmo e, assim, aumentar ainda mais a utilização do RSA \cite{burnett:02}.


Neste artigo serão apresentados brevemente os conceitos da criptografia, com ênfase na criptografia RSA. Para o seu entendimento foram detalhadas alguns procedimentos fundamentais para sua construção, como a geração de números primos grandes e o inverso modular, assim como o ataque por força bruta para a quebra das chaves.

O artigo está divido em seções, sendo que a primeira apresenta a criptografia, seguida da seção de metodologia de implementação, apresentando os procedimentos para obtenção da RSA e o ataque de força bruta, assim como os resultados dos testes, por fim a conclusão do trabalho.


\section{Criptografia} \label{sec:firstpage}

    \textbf{Sugestão de tópicos}
    \newline
    * Descrever em outas palavras a criptografia \newline
    * Descrever simétrica  \newline
    * Descrever assimétrica \newline
    * Descrever chave pública \newline
    * Desenho de chave pública \newline

Quando há necessidade de se proteger informações sigilosas utiliza-se o processo de criptografia para codificá-las de forma que não possam ser facilmente interpretadas. A leitura das informações criptografadas precisa passar por um novo processo de conversão, que é descriptografia, para que se torne novamente legível.

Um tipo muito comum de criptografia é chamado de chave simétrica. Nessa abordagem, um algoritmo utiliza uma chave para codificar informações em algo que se parece com um conjunto aleatório. O algoritmo, então, se utiliza da mesma chave para recuperar os dados originais \cite{burnett:02}.

Um sistema de criptografia de chave pública pode ser usado para codificar mensagens enviadas entre dois participantes de uma comunicação, de forma que um intruso que escute as mensagens codificadas não possa decodificá-las. Um sistema de criptografia de chave pública também permite que um dos participantes acrescente uma "assinatura digital"\ impossível de forjar ao final de uma mensagem eletrônica. Tal assinatura é a versão eletrônica de uma assinatura manuscrita em um documento de papel. Ela pode ser conferida com facilidade por qualquer pessoa, não pode ser forjada por ninguém, e ainda perde sua validade se qualquer bit da mensagem for alterado. Portanto, ela fornece autenticação, tanto da identidade do signatário quanto do conteúdo da mensagem assinada \cite{cormen:02}.

No sistema de criptografia RSA um usuário cria e torna pública uma chave baseada em dois números primos grandes, os quais são mantidos em segredo, ao mesmo tempo que cria uma chave privada, também mantida em segredo. Qualquer pessoa pode usar a chave pública para codificar uma mensagem e enviá-la ao destinatário que, em posse da chave privada, consegue descriptografá-la facilmente.

Cormen (2002) descreve os procedimentos a seguir, exemplificados pelos autores:
\begin{enumerate}
    \item Selecionar dois números primos grandes $p$ e $q$, sendo $p \neq q$, tal como 19 e 31;
    \item Calcular $n$ pela equação $n=p*q$, por exemplo $n$ = 589;
    \item Calcular o totiente de Euler, dado por $\phi(n) = (p-1)*(q-1)$, nesse caso $\phi(n) = 540$;
    \item Selecionar um inteiro ímpar pequeno $e$, tal que seja primo relativo de $\phi(n)$, satisfazendo a condição $MDC(e,\phi(n)) = 1$, nesse exemplo $e = 59$;
    \item Calcular $d$ como o inverso multiplicativo de $e$, utilizando o Algoritmo de Euclides estendido, que nesse exemplo resultaria em $d = 119$;
    \item O par formado por $P = (e,n)$ compõe a chave pública do RSA  e o par formado por $S = (d,n)$ a chave privada.
\end{enumerate}

O tamanho da chave é o fator mais importante para garantia da segurança no processo de criptografia. A Infraestrutura de Chaves Públicas do Brasil – ICP-Brasil recomenda o uso do algoritmo RSA como padrão para geração de chaves criptográficas de, no mínimo, 2048 bits \cite{icp:09}. A segurança do criptossistema RSA é fortemente vinculada à fatoração de $n$, que pode revelar os valores de $p$ e $q$. É devido ao grande esforço que demanda realizar essa fatoração e quebra da chave que se entende que o sistema de criptografia RSA é seguro. \cite{goodrich:13}.

\section{Metodologia}

    \textbf{Sugestão de tópicos}
    \newline
    Como foi construido o programa de criptografia

\subsection{Números primos grandes}

bla bla bla

\subsection{Algoritmo de Euclides}

bla bla bla

\subsection{Inverso Modular}

bla bla bla

\subsection{Força bruta}

bla bla bla


\section{Conclusão}

De modo algum a criptografia é a única ferramenta necessária para assegurar a segurança dos dados, nem resolverá todos os problemas de segurança. É um instrumento entre vários outros. Além disso, a criptografia não é à prova de falhas. Toda criptografia pode ser quebrada e, sobretudo, se for implementada incorretamente, ela não agrega nenhuma segurança real \cite{burnett:02}.



\bibliographystyle{sbc}
\bibliography{sbc-rsa}

\end{document}
