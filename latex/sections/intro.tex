A criptografia costuma ser definida como a arte de escrever em cifra ou em código, de modo a permitir que somente quem conheça o código possa ler a mensagem \cite{marcacini:10}. Com o crescente uso de dispositivos conectados a internet, como computadores, smartphones, equipamentos industriais (IIOT) e domésticos (IOT), tornam-se cada vez mais presentes na vida diária. Desta forma, a necessidade da segurança eletrônica torna-se ainda mais crítica. A utilização eficaz de técnicas criptográficas está no núcleo de várias dessas estratégias de gerenciamento de riscos de roubo de informações. \cite{burnett:02}


O sistema de criptografia de chave pública RSA se baseia na diferença drástica entre a facilidade de encontrar números primos grandes e a dificuldade de fatorar o produto de dois números primos grandes \cite{cormen:02}. A patente do algoritmo RSA expirou em setembro de 2000, o que permite que qualquer pessoa possa criar implementações desse algoritmo e, assim, aumentar ainda mais a utilização do RSA \cite{burnett:02}.


Neste artigo serão apresentados brevemente os conceitos da criptografia, com ênfase na criptografia RSA. Para o seu entendimento foram detalhadas alguns procedimentos fundamentais para sua construção, como a geração de números primos grandes e o inverso modular, assim como o ataque por força bruta para a quebra das chaves.

O artigo está divido em seções, sendo que a primeira apresenta a criptografia, seguida da seção de metodologia de implementação, apresentando os procedimentos para obtenção da RSA e o ataque de força bruta, assim como os resultados dos testes, por fim a conclusão do trabalho.
